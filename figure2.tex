%% Copyright 2022 Zijian Wang. All rights reserved.
%% Email: felix_wzj@yahoo.com
%% The newest version can be obtained from GitHub.

\documentclass[preprint]{elsarticle}

\hfuzz=2pt
%% \special{dvipdfmx:config z 0} %% speed boost

%% The commands below, learnt from TeX stack exchange can cancel the hbox overfull warning...
\usepackage{etoolbox}
\makeatletter
\patchcmd\ps@pprintTitle
{\fi\fi\fi\fi}
{\fi\fi\fi\fi
	\afterassignment\fix@elsarticle\let\@tempa}
{}{\FailedToPatch}
\def\fix@elsarticle{\iffalse{\fi}\romannumeral-`0}
\makeatother

%% \usepackage[letterpaper, total={5.3in, 9in}]{geometry}	%% set paper size & margin
\usepackage{amsmath}	%% math part 1
\usepackage{flexisym, breqn}	%% math part 2
\usepackage{amssymb, graphics, setspace}	%% math part 3
\usepackage{tikz, tikz-cd}	%% for commutative diagrams
\usepackage{flowchart}	%% for flowchart
\usepackage{float, booktabs}	%% for table & alignment
\usepackage{algorithmicx, algorithm, algpseudocode}	%% for pseudo code composition
\usepackage{blindtext}	%% overfull hbox solution
\usepackage{setspace}	%% double spacing

\usetikzlibrary{positioning, cd, arrows}	%% for comutative diagram & flowchart drawing

\doublespacing

\begin{document}
	\begin{figure}[ht]
		\centering
		\begin{tikzpicture}[commutative diagrams/every diagram]
			\node (K0) at (90:2.3cm) {$\mathcal{K}_0$};
			\node (K1) at (90+72:2cm) {$\mathcal{K}_1$};
			\node (K2) at (90+2*72:2cm) {$\mathcal{K}_2$};
			\node (K3) at (90+3*72:2cm) {$\mathcal{K}_3$};
			\node (Kn) at (90+4*72:2cm) {$\mathcal{K}_N$};
			
			\path[commutative diagrams/.cd, every arrow, every label]
			(K0) edge node {$k_0$} (K1)
			(K1) edge node {$K_1$} (K2)
			(K2) edge node {$k_2$} (K3)
			(K1) edge node {} (K3)
			(K1) edge node {} (Kn)
			(K0) edge node {} (K3);
			
			\path[commutative diagrams/.cd, every arrow, every label, dashed]
			(K3) edge node {$K_3$} (Kn)
			(Kn) edge node {} (K1)
			(Kn) edge node {$k_N$} (K0);
		\end{tikzpicture}
		\caption{An example map after several reinforcements.}
		\label{figure-2}
	\end{figure}
\end{document}